% !TeX encoding = UTF-8
% !TeX spellcheck = en_US
\documentclass[xcolor={svgnames},10pt,
%notes,
%handout
]{beamer}
\useoutertheme{metropolis}
\useinnertheme[sectionpage=progressbar,subsectionpage=progressbar]{metropolis}
\usefonttheme{metropolis}
\usecolortheme{seahorse} 
%\usecolortheme{dolphin} 
%\usecolortheme{metropolis}
\setbeamertemplate{note page}[plain]
\usepackage{lmodern}
\usepackage[utf8]{inputenc}
\usepackage[T1]{fontenc}
\graphicspath{{graphics/}}
\usepackage{csquotes}
\usepackage{listings}

\lstset{%
	language=R,                     % the language of the code
	basicstyle=\small\ttfamily, % the size of the fonts that are used for the code
%	numbers=left,                   % where to put the line-numbers
%	numberstyle=\tiny\color{DarkBlue},  % the style that is used for the line-numbers
	stepnumber=1,                   % the step between two line-numbers. If it is 1, each line	% will be numbered
	numbersep=5pt,                  % how far the line-numbers are from the code
%	backgroundcolor=\color{white},  % choose the background color. You must add \usepackage{color}
	showspaces=false,               % show spaces adding particular underscores
	showstringspaces=false,         % underline spaces within strings
	showtabs=false,                 % show tabs within strings adding particular underscores
	frame=single,                   % adds a frame around the code
	rulecolor=\color{black},        % if not set, the frame-color may be changed on line-breaks within not-black text(e.g. commens (green here))
	tabsize=2,                      % sets default tabsize to 2 spaces
	captionpos=b,                   % sets the caption-position to bottom
	breaklines=true,                % sets automatic line breaking
	breakatwhitespace=false,        % sets if automatic breaks should only happen at whitespace
%	keywordstyle=\color{DarkGreen},      % keyword style
	commentstyle=\color{DarkRed},   % comment style
%	stringstyle=\color{DarkGreen}      % string literal style
} 


\begin{document}

\title[Introduction to R]{Introduction to R}
\author{Dr.\ Willi Mutschler }
\date{~}
\institute{}
\maketitle

\section{Introduction}

\begin{frame}
\frametitle{Aims and prerequisites}
\begin{itemize}
\item Objective: Learn how to use R for econometrics and statistics
\item Prerequisites:
\begin{enumerate}
\item Basics in probability theory and statistical inference
\item Multiple linear regression
\end{enumerate}
\end{itemize}
\end{frame}


\begin{frame}
\frametitle{Literature}
\begin{description}
\item[Essential:] Andreas Behr and Ulrich P\"{o}tter (2011): \emph{Einf\"{u}hrung in die Statistik mit R}
\item[Additional:] Muenchen, Hilbe (2012): \emph{R for Stata Users}
\item Introductory courses on datacamp.com
\item Introductory tutorials on r-bloggers.com
\end{description}
\end{frame}

\begin{frame}
\frametitle{Topics}
\begin{enumerate}
\item What is R?
\item R-Studio Basics
\item Managing workspace and packages
\item Get help and understand the documentation
\item Programming language basics
\item Controlling functions
\item Data structures and acquisition
\item Selection and transformations of variables and observations
\item Treatment of missing values
\item Data visualization (basic and using gramar)
\item Basic statistics
\item Linear modeling: regression analysis
\end{enumerate}
\end{frame}


\section{What is R?}

\begin{frame}[standout]
\frametitle{What is R?}
\begin{itemize}
	\item[] \enquote{The most powerful statistical computing language on the planet...}\pause
	\item[] ...It all depends on the use and user
\end{itemize}
\end{frame}

\begin{frame}
\frametitle{What is R?}
\begin{itemize}
\item The language S (an object-oriented statistical computing language) is implemented as S-Plus (commercial) and R (OpenSource)
\item R is a
\begin{enumerate}
	\item language
	\item package
	\item environment
\end{enumerate}
for graphics and data analysis
\item R is FOSS (think about collaborations!) and easily extendable
\item Similar programming languages: Matlab, GAUSS, Julia
\end{itemize}
\end{frame}
\note{
The differences between S-Plus and R are minimal\\
}

\begin{frame}\frametitle{Comparison to STATA}
Basically five independent parts of a software
\begin{itemize}
\item Data input and management (language)
\item Statistics and graphics procedures / commands
\item Output management systems
\item Macro language
\item Matrix language
\end{itemize}
$\hookrightarrow$In other softwares, e.g. Stata, these are standalone and developed separately

$\hookrightarrow$In R all five were planed to be unified from the beginning
\end{frame}
\note{Output management systems: Stata return codes, postestimation, Matrix language in Stata is Mata}

\begin{frame}\frametitle{Comparison to STATA}
\begin{itemize}
\item Base R plus recommended packages contains over 1600 functions similar to e.g. STATA
\item Tested via extensive validation programs
\begin{itemize}
\item[$\hookrightarrow$] R is accurate even though no company behind it, R responds very quickly to bugs etc.
\item[$\hookrightarrow$] Source code of R (\textit{scripts}) are similar to \textit{Do Files} in STATA
\end{itemize}
\item Note that new statistical methods are nowadays often first published in R, and only later included by PROGRAMMERS (not original scientists) into STATA
\end{itemize}
\note{This has pro and cons}
\end{frame}

\begin{frame}[fragile]
\frametitle{R Console}\small
The basic command window is called the \texttt{R Console}\\
Prompt: \texttt{>}\\
You can input commands and execute them (by pressing the \textsc{Return} key)
\begin{lstlisting}
> 1+1
> 1+1 # This is a comment: 1+1
> (1+2)*3
> (5/3)^4.5
> 5+2; 7+3; 2*5
> pi
> Pi
> PI
> 2*((1+2)*(1-2)
\end{lstlisting}
This will: (1) evaluate it, (2) print the result, (3) count the lines with [n] and (4) delete the result
\end{frame}
\note{That a plus sign appears if we forget to close paranthesis.
	
Upper and lower cases, dot and comma: R differentiates between upper and lower cases, try:
pi, Pi und PI. Keep that in mind when calling functions and commands, and also when naming
variables. 

The decimal point is the dot.
}

\begin{frame}
\frametitle{Editors}
\begin{itemize}
	\item Long computations should not be done interactively in the command	window
	\item Use an editor to write a program and then execute it in R
	\item There is a built-in editor in R: \texttt{Datei -- Neues Skript}
	\item External editors:
	\begin{itemize}
		\item \textbf{R-Studio [RECOMMENDED!]}
		\item Tinn-R, Notepad++, Atom, Emacs, etc. are also possible
	\end{itemize}
\end{itemize}
\end{frame}

\begin{frame}
\frametitle{R-Studio}
\begin{itemize}
	\item Overview of four panels in R-Studio:
	\begin{enumerate}
		\item Script editor
		\item Environment|History|Connections,
		\item Console|Terminal
		\item Files|Plots|Packages|Help|Viewer
	\end{enumerate} 
	\item Important shortcuts (see also the Magic Wand)
	\begin{itemize}
		\item ~[CTRL+ENTER] for Run
		\item ~[CTRL+SHIFT+C] for commenting a section
		\item ~[CTRL+L] clears command windows
		\item ~[TAB] Function completion
		\item ~[ARROWS UP AND DOWN] in command window: scroll through history
		\item ~[F1] gets you help
	\end{itemize}
\end{itemize}
\end{frame}
\note{One can change this layout as one sees fit in the options interface

Actually R-Studio can also be run on a server easily such that one could call rstudio.mutschler.eu}


\begin{frame}[fragile]
\frametitle{Concatenation and Assignments}
Open a new Script in R-Studio and try out the following:
\begin{lstlisting}
c(1,4,7)
a <- c(1,4,7)
print(a)
a
A
b <- c(1,a,3)
(b <- c(1,a,3))
mean(b)
demo("graphics")
\end{lstlisting}
Execute a single line (or multiple lines by marking them) by pressing \textsc{CTRL-ENTER}
\end{frame}
\note{
Variables are case-sensitive, e.g. pi, Pi, PI

Assignments: evaluate expression and store result <- ("gets"), = works to but avoid it
you can enter command with semicolon or separate commands, but usually just new line

a + at the command line indicates that a command is not complete, either complete it or hit ESC, - if you did not complete commands, you get a +, mostly ), or hit ESC key or STRG/CMD+C

}


\section{Managing workspace}

\begin{frame}[fragile]
\frametitle{Managing workspace}
Listing Objects
\begin{itemize}
\item \lstinline|ls()| or \lstinline|objects()| lists the objects in your workspace
\item \lstinline|list=ls())| clears workspace
\end{itemize}
Working Directory
\begin{itemize}
	\item Easiest: Use GUI, i.e. Session - Set Working Directory
	\item alternatively: \lstinline|getwd()| and \lstinline|setwd("c:/temp")|
	\item Note that the path name is structured by slashes (\lstinline|/|), not backslashes (\lstinline|\|)
	\item some special hidden files are in your working directory
\end{itemize}
\end{frame}
\note{Special R Files: Rproj that remebmers the open files an working directory, stores settings in a folder named .Rproj.user
	R looks for the folowing files: .Rprofile (like profile.do) commands are automatically executed at startup (not a good practice)
	.Rhistory history of all statements! click on history tab, or history()
	.Rdata (not good)

SAve as
save.image(file="file.RData")
save(mydata,mylist,mymatrix,file="mypractice.RData")
Save WOrkspace image --> creates .RData in the current wd, when you start it will load it automatically, (TOOLS OPTION GENERAL, what you want)


}

\begin{frame}[fragile]
\frametitle{Managing workspace}
Quitting
\begin{itemize}
	\item Quit R by the command \lstinline|q()|
	\item Quit RStudio by using the GUI or [CTRL+Q]
	\item In general, do \emph{not} save your workspace
\end{itemize}
\end{frame}

\begin{frame}[fragile]
\frametitle{Managing workspace}
Misc
\begin{itemize}
\item if you did not complete commands, you get a \lstinline|+|. Most of the times close a \lstinline|)|, or hit ESC key or CTRL+C
\item comments go from \lstinline|#| to the end of line, can be between functions or in the middle with line breaks
\item there is no block comment features, simply use [CTRL-SHIFT-C] in R-Studio to (un)comment sections
\end{itemize}
\end{frame}
\note{
d <-   \#output name\\
2 *  \#first param\\
3    \#second param\\
print(d)
}

\section{Packages}

\begin{frame}
\frametitle{Packages}
\begin{itemize}
	\item One of the strengths of R is the large and growing collection of packages that can be downloaded from CRAN (or Github, etc)
	\item Installation (only once!)
	\begin{itemize}
	\item Use R-Studio interface for Packages
	\item \lstinline|install.packages("packagename")|
	\end{itemize}
	\item Installed packages are activated by \lstinline|library("packagename")|
	\item Help about packages: \lstinline|library(help="packagename")|
\end{itemize}
\end{frame}
\note{
Actually, library does not need quotations, but who cares.
}

\begin{frame}[fragile]
\frametitle{Common problems}
\begin{itemize}
\item \lstinline|install.packages("ggplot")|
	\begin{itemize}
	\item[$\hookrightarrow$] either not available or wrong package name 
	\end{itemize}
\item \lstinline|install.packages(ggplot2)|
	\begin{itemize}
	\item[$\hookrightarrow$] forgot quotes
	\end{itemize}
\item \lstinline|library("prettyR")|
	\begin{itemize}
	\item[$\hookrightarrow$] forgot to install it
	\end{itemize}
\item \lstinline|library("dplyr")|
	\begin{itemize}
	\item[$\hookrightarrow$] masked or covered up, that's okay
	\end{itemize}
\item \lstinline|detach("package:dplyr")| is opposite of library, which might prevent conflicts in function names and save on memory
\end{itemize}
\end{frame}

\begin{frame}
\frametitle{Which packages to use?}
\begin{itemize}
\item The Comprehensive R Archive Network (CRAN): can be searched by name or via Task Views for key programs on \texttt{cran.r-project.org}
\item \texttt{crantastic.org}: rated software
\item \texttt{rdocumentation.org}: top packages
\item \texttt{r-bloggers.com}
\item Git[hu|la]b, ...
\end{itemize}
\end{frame}

\begin{frame}[standout]
Exercise 1
\end{frame}
\section{Help and documentation}

\begin{frame}[fragile]
\frametitle{Help and documentation}
\begin{itemize}
	\item To obtain details about a command, type\\ \lstinline|?command| or \lstinline|help(command)|
\begin{lstlisting}
?mean
help(mean)
help("for")
help("while")
?"while"
help(package = "prettyR")
methods(plot) #gives you overview of extra functions, e.g. 
help(plot.lm)
help.start()
\end{lstlisting}
\item R-Studio: select/click on a command and hit F1
\end{itemize}
\end{frame}
\note{Some commands belong to a certain package or base, and need to be asked for with quotation marks}

\section{Programming Language Basics}

\begin{frame}[fragile]
\frametitle{R objects}
\begin{itemize}
\item R is object oriented
\item An object can be anything: scalar, vector, matrix, string, table, factor, list, data frame, regression results, model, \ldots
\item object name should begin with letter, no numbers, no underscores, no special characters, case matters
\item The object type determines how some commands work (e.g. \lstinline|plot|, \lstinline|summary|)
\item Every object has a unique name
\end{itemize}
\end{frame}

\begin{frame}[fragile]\frametitle{Parenthesis}
(Parenthesis)
\begin{itemize}
\item control math order as usual in algebra, e.g.
\begin{lstlisting}
1+1*10
(1+1)*10
\end{lstlisting}
\item print assignment values: 
\begin{lstlisting}
(x<-12)
\end{lstlisting}
\item provide options to functions, e.g.
\begin{lstlisting}
mean(x, na.rm=FALSE)
\end{lstlisting}
\end{itemize}
\end{frame}

\begin{frame}[fragile]\frametitle{Parenthesis}
\{Curly Braces\}
\begin{itemize}
\item Can combine many commands into one
\item Executes all assignments but returns only value of last expression
\item Useful for writing own functions
\begin{lstlisting}
{x<-2; y<-1
z<-x+y; z2<- z^2
z
z2}
\end{lstlisting}
\end{itemize}
[Square Braces] and [[Double Square Braces]]
\begin{itemize}
\item Used for selecting/indexing elements within objects
\item Double squares are used in lists (one of the most general data structure)
\end{itemize}
\end{frame}


\begin{frame}\frametitle{Variables}
All kinds of values can be stored in a variable (as we are object-oriented):
\begin{itemize}
\item numbers
\item letters
\item words
\item dates
\item logical TRUE/FALSE values
\item data structures
\item ....
\end{itemize}
\end{frame}


\begin{frame}[fragile]\frametitle{Mode, Class, Dimension and Length of Vectors}
\begin{lstlisting}
x <- c(1, 2, 1, 2, 1, 2, 1, 2)
print(x)
mode(x)
class(x)
length(x)
dim(x)
x + x
2*x
x + 10
x + c(10, 100)
\end{lstlisting}\pause
\begin{itemize}
	\item \lstinline|mode|: a variable's type
	\item \lstinline|class|: vectors have a class of character or numeric (or many other things), dimension (\lstinline|dim|) of a vector is NULL
	\item \lstinline|length|: number of elements it contains (including (!) missings)
	\item Note: If one vector is shorter, its values are \textbf{recycled} until the lengths match
\end{itemize}
\end{frame}
\note{Advantage: saves you typing,

Attention: a vector is NOT a 1D-vector!
}

\begin{frame}\frametitle{Operators and functions}
Logical operators
\begin{description}
\item[\texttt{\&}] and
\item[\texttt{|}] or
\item[\texttt{!}] not
\item[\texttt{NA}] not available or no answer
\item[\texttt{==}] equal (do \emph{not} use \texttt{=})
\item[\texttt{>}, \texttt{>=}] greater than,
greater than or equal
\item[\texttt{<}, \texttt{<=}] less than, less
than or equal
\item[\texttt{!=}] not equal
\end{description}
\end{frame}


\begin{frame}[fragile]
\frametitle{Operators and functions}
Examples logical operators
\begin{lstlisting}
5 < 7
1+1 == 3
a <- c(-1,4,9)
a >= 2 \& a < 8
b <- c(NA,1,2,3)
b > 0
is.na(b)
a[a>2]
a == 4
a = 4
\end{lstlisting}

\end{frame}

\begin{frame}[standout]
Exercise 2
\end{frame}


\begin{frame}\frametitle{Operators and functions}
Arithmetic operators and mathematical functions
\begin{description}
\item[\texttt{+}, \texttt{-}] plus, minus
\item[\texttt{*}, \texttt{/}] multiplication and division
\item[\texttt{\symbol{94}}] power (exponentiation)
\item[\texttt{Inf}, \texttt{-Inf}] infinity (plus or minus)
\item[\texttt{NaN}] not a number
\item[\texttt{abs}] absolute value
\item[\texttt{sqrt}] square root
\item[\texttt{exp,log}] exponential function and natural logarithm (\emph{not} \texttt{ln})
\item[\texttt{sin}] sinus (other trigonometric functions as well)
\item[\texttt{sum}] sum
\end{description}
\end{frame}


\begin{frame}[fragile]
\frametitle{Operators and functions}
Examples arithmetic operators and mathematical functions
\begin{lstlisting}
x <- c(-1,0,2,9,3)
abs(x)
sqrt(x)
1/x
-1/x
0/x
log(x)
x^c(2,3,2,3,2)
x^c(2,3)
log(x)<0
\end{lstlisting}
\end{frame}


\begin{frame}[standout]
Exercise 3
\end{frame}


\begin{frame}
\frametitle{Operators and functions}
Matrix functions
\begin{description}
\item[\texttt{matrix}] creates a matrix from a vector
\item[\texttt{dim}] dimensions of a matrix
\item[\texttt{t}] transpose
\item[\texttt{\%*\%}] matrix multiplication
\item[\texttt{det}] determinant
\item[\texttt{solve}] inverse
\item[\texttt{eigen}] eigenvalues and eigenvectors
\item[\texttt{diag}] diagonal
\item[\texttt{cbind}] merge matrices column-wise
\item[\texttt{rbind}] merge matrices row-wise
\end{description}
\end{frame}


\begin{frame}[fragile]
\frametitle{Operators and functions}
Examples matrix functions
\begin{lstlisting}
X <- matrix(c(2,3,4,5,1,1,9,3,2),3,3)
X
dim(X)
det(X)
solve(t(X)%*%X)
X*c(8,5,1)
X%*%c(8,5,1)
diag(X)
diag(X) <- 0
X
solve(X)%*%X
matrix(NA,4,4)
rbind(cbind(X,X),c(0,1))
\end{lstlisting}
Note the difference between \texttt{*} and \texttt{\%*\%}!
\end{frame}

\begin{frame}[standout]
Exercise 4
\end{frame}


\begin{frame}
\frametitle{Operators and functions}
Set operations and special functions
\begin{description}
\item[\texttt{unique}] the set of all unique elements of a vector
\item[\texttt{union}] $x\cup y$
\item[\texttt{intersect}] $x\cap y$
\item[\texttt{setdiff}] $x\backslash y$
\item[\texttt{\%in\%}] $x\in y$
\item[\texttt{sort}] sort the elements of a vector
\item[\texttt{cumsum}] cumulated sum of a vector \newline
(also \texttt{cumprod}, \texttt{cummin}, \texttt{cummax})
\item[\texttt{which(...)}] find the index of the vector element for which some condition is true
\item[\texttt{which.min}] find the index of the smallest vector element 
\newline
(also \texttt{which.max})
\end{description}
\end{frame}

\begin{frame}[standout]
Exercise 5
\end{frame}


\begin{frame}
\frametitle{Operators and functions}
Sequences and replications
\begin{description}
\item[\texttt{seq}] sequence from $a$ to $b$ of length $n$, \newline
\texttt{seq(from=a,to=b,length=n)}, \newline
or by increments of size $d,$ \newline
\texttt{seq(from=a,to=b,by=d)}
\item[\texttt{a:b}] integer sequence from $a$ to $b$
\item[\texttt{rep}] replicate a vector $n$ times\newline
\texttt{rep(what,times=n)},\newline
or each element $n$ times,\newline
\texttt{rep(what,each=n)}
\end{description}
\end{frame}

\begin{frame}[standout]
Exercise 6
\end{frame}


\section{Controlling Functions}

\begin{frame}[fragile]\frametitle{Calling Functions}
\begin{itemize}
	\item R is controlled by functions: when you use an R function you call it and pass values to their arguments
	\item arguments are listed in (parenthesis) and are separated by comas
	\item argument values are usually single objects and have a unique name
	\item function calls \textbf{return} a \textbf{value} (in help file, output is often called return/value)
	\item value is a single object, may contain much information, optimized for further analysis not (necessarily) for displaying
\end{itemize}
\end{frame}
\note{Hint: dont abreviate TRUE AND FALSE by T and F. So be careful\\
note that we may use nested functions}

\begin{frame}[fragile]
\frametitle{More on function arguments}
Common Error: 
\begin{lstlisting}
x1 <-c(1,2,3); x2 <-c(5,6,7); x3 <-c(8,9,0);
mean(x1,x2,x3)                #nope!
summary(data.frame(x1,x2,x3)) #better!
\end{lstlisting}
\begin{itemize}
	\item When calling a function, the order of the arguments is arbitrary, if the argument names are explicitly used:\\
	\lstinline|mean(na.rm=FALSE,trim=.1,x=mydata)|
	\item Without argument names, R assigns the values in the order of the function definition:\\
	\lstinline|mean(mydata,.1,FALSE)|
	\item A function definition may include default values for arguments, e.g.\\
	\lstinline|mean(x, trim = 0, na.rm = FALSE)|
	\item If an argument with a default value is missing in a function call, R uses the default value
\end{itemize}
\end{frame}

\section{Data Structures}
\begin{frame}\frametitle{Data Structures}
Most Statistics programs have only one data structure, R is more flexible
\begin{itemize}
	\item Factors
	\item Arrays
	\item Vectors
	\item Matrices
	\item Data frames
	\item Tables 
	\item Lists
	\item or make your own
\end{itemize}
\end{frame}

\begin{frame}[fragile]\frametitle{Factors}
Does this make sense?
\begin{lstlisting}
degree <- c(0,     2,   1,  2,   3,   2,   1,   3)
gender <- c("f", "f", "f", NA, "m", "m", "m", "m")
degree[gender=="f"]
degree[gender=="m"]
table(degree)
table(gender)
summary(gender)
summary(degree)
summary(degree[gender=="m"])
\end{lstlisting}\pause
\begin{itemize}
\item No! We need to tell R that these are categorial variables! This is important as this will 
\begin{itemize}
\item print the right statistics and summaries
\item automatically include dummy variables in regression model
\end{itemize}
\item Note that NA is always included
\end{itemize}
\end{frame}
\note{No statistics, bar charts etc.; Es sind noch keine categorial variables festgelegt

Character elements must be in quotes, missing value \lstinline|NA| is not in quotes

NA's is displayed! As I don't know, I keep it.

R output is typically very sparse, optimized for further processing
}


\begin{frame}[fragile]\frametitle{Factors}
Better:
\begin{lstlisting}
degree <- factor(degree)
degree
summary(degree)
degree <- factor(degree, 
		  levels=c(1,2,3,4),
		  labels = c("BA","MA","PhD","Other"))
degree
summary(degree)
gender <- factor(gender, 
		  levels = c("m","f"),
		  labels = c("Male","Female"))
summary(gender)
degree[gender=="Female"] 
degree[gender=="f"] #note this does not work anymore!
\end{lstlisting}
Note that values you do not include in \lstinline|levels| become \lstinline|NA|
\end{frame}

\note{Values do not have to actually occur, if for instance you collect data in different sources, but you still want to labe. }

\begin{frame}\frametitle{Data Frames}
Why use data frames?
\begin{itemize}
\item data frames are rectangular set of variables
\item variables are called components (vectors, factors, columns)
\item observations are called rows or cases
\item mode is list, class is data.frame, components must have equal length (same number of observations)
\item variable names and row names are stored as attributes
\item Almost never required, but...
\begin{itemize}
\item lock values of observations together
\item ensures proper sorting
\item ensures correct NA removal
\end{itemize}
\end{itemize}
\end{frame}

\begin{frame}[fragile]\frametitle{Data Frames}\footnotesize
\begin{lstlisting}
testscores <- c("1.7", "1.3", "1.0", "1.7", "2.0")
mydata <- data.frame(degree, gender, testscores)
testscores <- c("1.7", "1.3", "1.0", "1.7", "2.0", NA, NA, NA)
mydata <- data.frame(degree, gender, testscores)
mydata
names(mydata)
rownames(mydata)
rownames(mydata) <- c("Bart","Homer","Maggie","Marge","Nelson","Apu","Moe","Krusty")
rownames(mydata)
mydata
class(mydata$testscores)
mydata$testscores <- as.numeric(as.character(mydata$testscores))
class(mydata$testscores)
\end{lstlisting}
\begin{itemize}
\item \lstinline|data.frame| converts character variables to factors unless you add \lstinline|stringsAsFactor = FALSE|
\end{itemize}
\end{frame}
\note{
Changing classes
as. coerce objects to change class temporarily when possible:
as.vector, as.character, as.numeric, as.factor, as.data.frame, as.matrix, as.list
}


\begin{frame}[fragile]
\frametitle{Large Data Frames}
For large data frames use \texttt{tbl\_df} from \texttt{dplyr} package
\begin{itemize}
	\item offers better printing of large data frames
	\item reports number of [rows by columns]
	\item prints only 10 observations (option can be changed)
	\item prints only enough variables to fill your screen
	\item class becomes tbl\_df, tbl, data.frame
	\item affects \lstinline|print()|
\end{itemize}
\begin{lstlisting}
data(Titanic)
detach("package:dplyr")
print(data.frame(Titanic))
plot(data.frame(Titanic))
library(dplyr)
print(tbl_df(Titanic))
plot(Titanic)
detach("package:dplyr")
\end{lstlisting}
\end{frame}


\begin{frame}\frametitle{Data Structures Overview}
Matrix
\begin{itemize}
\item same as data frame, but mode must be the same, i.e. atomic objects (all numeric or all character)
\item class is matrix
\item actually one long vector stored with a dimension attribute (dim)
\end{itemize}
Array
\begin{itemize}
\item Matrix that may have more than two dimensions
\item Vectors are 1D Arrays, Matrices are 2D Arrays
\item actually one long vector stored with a dimension attribute (dim)
\end{itemize}
\end{frame}

\begin{frame}[fragile]\frametitle{Data Structures Overview}
Lists
\begin{itemize}
\item object that can store any other type of objects, called components
\item \texttt{mylist <- list(name1 = comp1, name2 = comp2, ...)}
\begin{lstlisting}
mylist <- list(degree, gender, testscores)
mylist
mylist <- list(UniversityDegree=degree,
			   sex=gender,
			   "Test Score"=testscores)
mylist
names(mylist)
identical(mylist[[1]], mylist$UniversityDegree)
identical(mylist[[2]], mylist$sex)
identical(mylist[[3]], mylist$`Test Score`)
\end{lstlisting}
\item modeling functions often output their values as lists
\item for indexing we need double square brackets or names with \$ sign
\end{itemize}
\end{frame}

\begin{frame}\frametitle{Data Structures Overview}
Some useful commands
\begin{itemize}
\item \lstinline|print|
\item \lstinline|head|
\item \lstinline|tail|
\item \lstinline|names|
\item \lstinline|rownames|
\item \lstinline|mode|
\item \lstinline|class|
\item \lstinline|attributes|
\item \lstinline|str|
\end{itemize}

\end{frame}


\begin{frame}[fragile]
\frametitle{Sorting and merging}
\framesubtitle{Sorting}
\begin{itemize}
	\item The \lstinline|sort| command sorts (numeric or character) vectors
	\item By default, the elements are sorted ascendingly, but one can also sort descendingly.
	\item Matrices are sorted as vectors
	\item Dataframes cannot be sorted by \lstinline|sort|
	\item The function \lstinline|order(x)| returns a vector of the position of the smallest, the second smallest, \ldots , the largest elements of \lstinline|x|
	\item Hence, \lstinline|x[order(x)]| returns the sorted vector
	\item The \lstinline|order| command is useful for sorting matrices and dataframes!
\end{itemize}
\end{frame}


\begin{frame}
\frametitle{Sorting and merging}
\framesubtitle{Merging}
\begin{itemize}
\item Two dataframes can be merged by common column names
\item The command \lstinline|merge(x,y,by=...)| merges two dataframes \lstinline|x|
and \lstinline|x| by a common variable given in the \lstinline|by|-option
\item What happens if there are observations in \lstinline|x| that are missing in \lstinline|y| (or vice versa)?
\item There are options to choose the way R deals with missings
\end{itemize}
\end{frame}


\section{Data import and export}
\begin{frame}
\frametitle{Data import and export}
General remarks
\begin{itemize}
\item R is all about working with data
\item There are various ways to read data from different sources in many
formats
\item In R, datasets are usually represented as \lstinline|data.frame| objects
\item R has a large collection of \textquotedblleft standard datasets\textquotedblright , see \lstinline|data()|
\end{itemize}
\end{frame}

\begin{frame}
\frametitle{Data import and export - Manual data input}
\begin{itemize}
\item Very small datasets can be typed in directly, e.g.\newline
\lstinline|x <- data.frame(v1=c(2,6,1,1),v2=c(9,9,8,8))|
\item To edit existing objects, use \texttt{data.entry}, e.g.\newline
\texttt{y <- data.entry(x)}
\item However, editing data within R is \emph{not} recommended
\item Datasets should be stored outside R, preferably in separate directories
\item The datasets should be easily accessible by data-managing programs
(e.g. Excel, Stata, ASCII editors, \ldots )
\end{itemize}
\end{frame}

\begin{frame}
\frametitle{Data import and export - Saving and loading R objects}
\begin{itemize}
\item All R objects can be saved by the command\newline
\texttt{save(obj1,obj2,...,file="c:/path/name.Rdata")}
\item In principle, other file name extensions are possible, \newline
but not recommended
\item All objects saved in a file can be loaded by the command\newline
\texttt{load("c:/path/name.Rdata")}
\item The data format is R specific
\end{itemize}
\end{frame}

\begin{frame}
\frametitle{Data import and export - Reading and writing text files}
\begin{itemize}
\item A convenient command to read simple text files is \newline
\texttt{read.csv("c:/path/filename.txt")}
\item The command assumes the following data format:
\begin{enumerate}
\item The first row contains the variables names, delimited by commas
\item The following rows are the observations, the variables are again
delimited by commas
\item The decimal sign is a dot (not a comma)
\end{enumerate}
\item Use \texttt{read.csv2} if the variables are delimited by semi-colons
and the decimal sign is a comma (i.e. German style)
\item More options are available for the command \texttt{read.table}
\item Exporting text files from R is usually not necessary. If it is, use 
\texttt{write.csv}, \texttt{write.csv2} or \texttt{write.table}
\end{itemize}
\end{frame}

\begin{frame}[standout]
Exercise 7
\end{frame}

\begin{frame}[fragile]
\frametitle{Data import and export - Other data formats}
\begin{itemize}
\item there are many packages that provide easy access to datasets in other data formats
\item flat files
\begin{itemize}
\item \lstinline|readr|: fast, easy to use, consistent
\begin{itemize}
	\item \lstinline|read_delim| instead of \lstinline|read.table|
	\item \lstinline|read_csv| instead of \lstinline|read.csv|
	\item \lstinline|read_tsv| instead of \lstinline|read.delim|
\end{itemize} 
\item \lstinline|data.table| for huge data sets
\begin{itemize}
	\item \lstinline|fread| just works and ridiculously fast (infers column types and separators)
\end{itemize}
\end{itemize}
\item Excel
\begin{itemize}
	\item \lstinline|readxl| is fast
	\begin{itemize}
		\item \lstinline|read_excel("data.xlsx", sheet = "my_sheet")|
	\end{itemize}
	\item \lstinline|XLConnect| to have much more control and bridge Excel into R
	\item several other packages, e.g. \lstinline|gdata| uses Perl, \lstinline|xlsx| uses Java...
\end{itemize}
\item Databases
\begin{itemize}
	\item \lstinline|dbConnect| from \lstinline|DBI| package
\end{itemize}
\end{itemize}
\end{frame}
\note{XLConnect is for big companies... installing is a bit hard, due to java, }

\begin{frame}[fragile]
\frametitle{Data import and export - Other Statistical Software Packages}
\lstinline|haven|
\begin{itemize}
\item consistent, easy and fast (uses C library)
\item SAS (\lstinline|read_sas|), STATA (\lstinline|read_stata| or \lstinline|read_dta|), and SPSS (\lstinline|read_por| or \lstinline|read_sav|)
\end{itemize}
\lstinline|foreign|
\begin{itemize}
	\item less consistent, very comprehensive (saves everything into attributes),
	\item for formats \texttt{dbf}, \texttt{Stata}, \texttt{SPSS}, \texttt{SAS}, and a few more (but not Excel)
	\item \lstinline|read.dta| takes also care of STATAs different missing values
\end{itemize}

\end{frame}
\note{
haven: Stata 13 and 14

foreign: STATA 5 to 12, , might have to use saveold

\lstinline|Hmisc|: \lstinline|stata.get| for STATA with saveold}

\begin{frame}[standout]
Exercises 8, 9, 10 and 11
\end{frame}


\section{Selection and Transformations of Variables}

\begin{frame}\frametitle{Selecting Variables}
Most programming packages:
\begin{itemize}
	\item Select variables by name
	\item Select observations by logical condition
\end{itemize}
R can do that as well, but has many more ways to select and transform variables (we can even reverse this order)
\end{frame}


\begin{frame}
\frametitle{Indexing}
Indexing vectors
\begin{itemize}
	\item R has a rich indexing syntax
	\item The basic ideas are the same for vectors, matrices and other objects
	\item Indexing is used to read or manipulate specified elements of the
	objects
	\item Indexes are always given in square brackets: \texttt{[]} \newline
	(or sometimes \texttt{[[]]})
	\item Indexes can be either numerical or logical
	\item We will start with vectors and then look at matrices and dataframes
	\item The symbols \texttt{i} and \texttt{j} denote integer variables (not
	vectors)
\end{itemize}
\end{frame}


\begin{frame}
\frametitle{Indexing Vectors}
Numerical indexing
\begin{description}
\item[{\texttt{x[1]}}] first element
\item[{\texttt{x[2]}}] second element
\item[{\texttt{x[i]}}] $i$-th element
\item[{\texttt{x[-i]}}] all elements, without position $i$
\item[{\texttt{x[a:b]}}] all elements from position $a$ to position $b$
\item[{\texttt{x[k]}}] \texttt{k} numerical vector: all elements at positions
given in $k$
\end{description}

Logical indexing
\begin{description}
\item[{\texttt{x[a]}}] \texttt{a} logical vector: all elements where $a$ is
true \newline
(\texttt{a} must have the same length as \texttt{x})
\end{description}
\end{frame}


\begin{frame}[fragile]
\frametitle{Indexing Vectors}
Indexing vectors
\begin{lstlisting}
x <- c(2,3,4,5,1,1,9,3,2)
x[2]
x[4:7]
x[20]
x[-9]
x[-3]
x[c(1,5,1,9,9)]
a <- (x<4)
x[a]
x[x<4]
\end{lstlisting}
\end{frame}

\begin{frame}[standout]
Exercise 12
\end{frame}


\begin{frame}
\frametitle{Indexing Matrices}
Numerical indexing
\begin{description}
\item[{\texttt{x[i,j]}}] element in row $i$, column $j$
\item[{\texttt{x[,j]}}] column $j$ (as a vector)
\item[{\texttt{x[i,]}}] row $i$ (as a vector)
\item[{\texttt{x[,-j]}}] without column $j$
\item[{\texttt{x[-i,]}}] without row $i$
\item[{\texttt{x[a:b,j]}}] elements $a$ to $b$ in column $j$
\item[{\texttt{x[k,m]}}] \texttt{k},\texttt{m} numerical vectors: all
elements at positions \newline
given in $k$ and $m$
\end{description}
\end{frame}


\begin{frame}
\frametitle{Indexing Matrices}
Logical indexing
Let \texttt{a} denote a logical matrix of the same dimension as \texttt{x}; 
\newline
let \texttt{k} and \texttt{m} denote logical vectors of suitable length
\begin{description}
\item[{\texttt{x[a]}}] All elements of \texttt{x} at positions where \texttt{a%
} is true, \newline
as a \emph{vector!}
\item[{\texttt{x[,m]}}] All columns of \texttt{x} where \texttt{m} is true
\item[{\texttt{x[k,]}}] All rows of \texttt{x} where \texttt{k} is true
\end{description}
Of course, one may use numerical indexing for one dimension \newline
and logical indexing for the other dimension
\begin{description}
\item[{\texttt{x[k,1:2]}}] All elements of columns 1 and 2 where \texttt{k}
is true
\item[{\texttt{x[3,m]}}] All elements of row 3 where \texttt{m} is true
\end{description}
\end{frame}


\begin{frame}[fragile]
\frametitle{Indexing Matrices}
Examples
\begin{lstlisting}
x <- matrix(1:16,4,4)
x[3,3]
x[,4]
x[2,]
x[,-1]
x[-3,]
x[2:4,4]
x[c(1,4,2,2,2),1:2]
\end{lstlisting}
\end{frame}


\begin{frame}[fragile]
\frametitle{Indexing Matrices}
Examples
\begin{lstlisting}
x <- matrix((-7:8)^2,4,4)
a <- (x<10)
x[a]
x[,c(TRUE,FALSE,TRUE,FALSE)]
x[x[,1]<30,3:4]
x[x[,2]==1 | x[,3]==1,]
x[2:4,4]
x[c(1,4,2,2,2),1:2]
\end{lstlisting}
\end{frame}

\begin{frame}[standout]
Exercise 13
\end{frame}


\begin{frame}\frametitle{Indexing Data Frames}
\begin{itemize}
\item Dataframes have the same index methods as matrices
\item Logical conditions can include strings (character variables)
\item There are three additional ways to extract data frame columns:
\begin{enumerate}
\item \texttt{x\$varname}
\item \texttt{x[[i]]} \newline
where \texttt{i} can also be a numerical vector
\item \texttt{x["varname"]} \newline
or \texttt{x[c("varname1","varname2",...)]}
\end{enumerate}
\item Dataframe variables can be addressed directly by their name when you 
\texttt{attach} the dataframe, e.g. \texttt{attach(x)}
\item \texttt{all}, \texttt{any} and \texttt{which} are also useful here
\end{itemize}
\end{frame}
\note{you can use the attach function, copies the variable into a temp space that R searches after your workspace they're available as vectors and factors Eventually detach(mydata) is optional to free up memory space, as this is only a copy!

	with(mydata,summary(q1))
	now R looks in data frame BEFORE workspace
	but you need to do this for all commands (different to attach)
}

\begin{frame}[fragile]
\frametitle{Indexing Data Frames}
Common Error

\begin{lstlisting}
mean(mydata["testscore"])  #will give you NA
mean(mydata[,"testscore"]) #Don't forget the comma!
\end{lstlisting}
\end{frame}

\begin{frame}[standout]
Exercise 14
\end{frame}



\begin{frame}[fragile]\frametitle{select from dplyr package}
\begin{itemize}
\item the \lstinline|select| function makes life much easier as it selects all kinds of variables and always returns a data frame
\begin{lstlisting}
library(dplyr)
select(mydata100, degree, gender) # for as many as I like
select(mydata100, gender:q4) # take all variables that are in between and itselft too
select(mydata100, contains("q"))
select(mydata100, starts_with("q"))
select(mydata100, ends_with("shop"))
select(mydata100, num_range("q", 1:4))
\end{lstlisting}
\item you can also use regular expressions
\item be careful, most stat functions work on vectors, not on data frames
\end{itemize}
\end{frame}


\begin{frame}[fragile]
\frametitle{Selecting observations}
Put logic in the row position (before the comma)
\begin{lstlisting}
summary(mydata100[mydata100$gender == "f", ]) #don't forget the comma!
\end{lstlisting}
you could actually index on different objects
\end{frame}

\begin{frame}[fragile]
\frametitle{filter function from dplyr package}
Use the filter() function
\begin{lstlisting}
library("dplyr")
summary(filter(mydata100, gender == "f"))
\end{lstlisting}
\end{frame}

\begin{frame}[fragile]
\frametitle{Selecting both variables and observations}
Traditional way:
\begin{lstlisting}
myVars <- c("gender", "q1", "q2", "q3", "q4")
myObs <- which(mydata100$gender == "f")
mysubset <- mydata100[myObs, myVars]
summary(mysubset)
\end{lstlisting}
Modern way:
\begin{lstlisting}
library("dplyr")
mysubset <- select(mydata100, gender, q1:q4)
mysubset <- filter(mysubset, gender == "f")
summary(mysubset)
\end{lstlisting}
\begin{itemize}
\item 
first call select or filter, whichever gives you the smallest subset
\item advanced feature "pipes" \%>\%: feeds results from one to another
\end{itemize}
\end{frame}


\begin{frame}[fragile]\frametitle{Transformations}
Very tedious:
\begin{lstlisting}
mydata2[, "diff"]  <- mydata[, "q4"] - mydata[, "q1"]
mydata2[, "ratio"] <- mydata[, "q4"] / mydata[, "q1"]
mydata2[, "q4log"] <- log(mydata[, "q4"])
mydata2[, "z4"] <- as.numeric(scale(mydata[, "q4"]))
mydata2[, "meanQ"] <- (mydata[, "q1"] + mydata[, "q2"] + mydata[, "q3"] + mydata[, "q4"])/4
\end{lstlisting}
Much cooler: \lstinline|mutate| function from \lstinline|dplyr|
\begin{lstlisting}
mydata2 <- mutate(mydata,
	diff = q4 - q1,
	ratio = q4 / q1,
	q4log = log(q4),
	z4 = scale(q4),
	meanQ = (q1+q2+q3+q4)/4
)
mydata2
\end{lstlisting}
\begin{itemize}
\item lets you use short names
\item returns original data frame plus the new variables for every row
\item works only on columns or variables
\end{itemize}
\end{frame}
\note{Making a copy to not overwrite dataset, don't forget commas!}

\begin{frame}[standout]
Exercise 15
\end{frame}


\section{Graphics}

\begin{frame}[fragile]\frametitle{Some Remarks on Graphics}
Traditional or Base Graphics
\begin{itemize}
\item \lstinline|plot()| offers many methods
\item extremely flexible and extensible, but not easy to use with groups
\item Uses \enquote{traditional graphics system}
\end{itemize}
\begin{lstlisting}
load("mydata100.RData")
mydata100 <- na.omit(mydata100)
attach(mydata100)
head(mydata100)
plot(workshop)
plot(workshop,gender)
plot(gender,workshop)
plot(workshop,posttest)
plot(posttest,workshop)
plot(posttest)
plot(pretest,posttest)
hist(posttest)
rug(posttest)
\end{lstlisting}
\end{frame}
\note{plots of factors, these are standardized rather than counts

plot(posttest,workshop) is a strip plot
}

\begin{frame}[fragile]\frametitle{Many options}
Nicer plots
\begin{lstlisting}
plot(pretest, posttest
	pch = 19, # plot character
	cex = 2,  # character expansion
	main = "My Main Title",
	xlab = "My X Axis Label",
	ylab = "My Y Axis Label")
grid() #add grid to plot
par() #graphics parameters
\end{lstlisting}
\end{frame}

\begin{frame}[fragile]\frametitle{Subplots}
\begin{lstlisting}
par(mfrow = c(2, 1))  # 2 rows, 1 column
plot(workshop[gender == "Female"], main = "The Females")
plot(workshop[gender == "Male"], main = "The Males")
#scatter plot with regression line
par(mfrow = c(1, 1))
plot(pretest,posttest)
abline(c(18.78,0.845))
myModel <- lm(posttest ~ pretest, data=mydata100)
abline(coefficients(myModel))
\end{lstlisting}
Problems
\begin{itemize}
\item axis are not standardized
\item a lot of white unnecessary space
\item titles of each plot are in the vertical position taking valuable space
\item one could fix all this with several options...
\end{itemize}
\end{frame}

\begin{frame}[fragile]\frametitle{ggplot - Basic idea}
\lstinline|ggplot|
\begin{itemize}
	\item follows Wilkinson's Grammar of Graphics
	\item works with underlying graphics concepts, not pre-defined graph types
	\item enables to create any data graphic that you can think of
	\item uses Grid Graphics System instead of traditional system
\end{itemize}
Grammar of Graphics
\begin{itemize}
\item Asthetics: how will variables appear? On axis? Shape, color, size...
\item Geoms: set geometric objects, e.g. points, bars, lines, boxes
\item Statistics: bins, smoother, fits,...
\item Scales: axes (regular, log), legends
\item Coordinate system: cartesian or polar
\item Facets: plot by group(s)
\end{itemize}
\end{frame}
\note{how to create any kind of graphs}


\begin{frame}[fragile]\frametitle{ggplot - examples}
\begin{lstlisting}
library("ggplot2")
ggplot(mydata100, aes(workshop)) +
	geom_bar()
	
ggplot(mydata100, aes(workshop), fill = gender) +
	geom_bar(position="stack")
	
ggplot(mydata100, aes(workshop), fill = gender) +
	geom_bar(position="stack") +
	scale_fill_grey()
	
ggplot(mydata100, aes(workshop), fill = gender) +
	geom_bar(position="dodge")
	
ggplot(mydata100, aes(workshop), fill = gender) +
	geom_bar() +
	facet_grid(gender ~ .)
\end{lstlisting}
\end{frame}
\note{aes: asthetic, with very nice defaults

As an argument to this function, you specify a formula, where the factor before the \~ specifies the number of rows, while the factor after the \~ specifies the number of columns.
}

\begin{frame}[fragile]\frametitle{ggplot - examples}

\begin{lstlisting}[basicstyle=\footnotesize]
ggplot(mydata100, aes(workshop, posttest)) +
	geom_boxplot() +
	geom_point()
ggplot(mydata100, aes(workshop, posttest)) +
	geom_boxplot() +
	geom_point() + 
	facet_grid(. ~ gender)	
ggplot(mydata100, aes(pretest,posttest)) + 
	geom_points()	
ggplot(mydata100, aes(pretest,posttest, shape = gender)) + 
	geom_points(size = 5)	
ggplot(mydata100, aes(pretest,posttest, shape = gender, linetype = gender)) + 
	geom_points(size = 5) +
	geom_smooth(method = "lm")	
ggplot(mydata100, aes(pretest,posttest, shape = gender, linetype = gender)) + 
	geom_point() +
	geom_smooth(method = "lm") + 
	facet_grid(workshop ~ gender)
\end{lstlisting}

\end{frame}
\note{do analysis how equal are slopes, you get a graphic on how unequal the slopes are, you can check whether the ranges are similar
\begin{itemize}
	\item layer will compe on top
	\item there are no rows but gender in the columns
\end{itemize}
}

\begin{frame}[fragile]\frametitle{ggplot - examples}
Same, more fully specified
\begin{lstlisting}
ggplot() + 
	geom_point(data=mydata100, aes(pretest, posttest, shape = gender)) +
	geom_smooth(data=mydata100, aes(pretest, posttest), method = "lm") +
	facet_grid(workshop~gender)
\end{lstlisting}
\begin{itemize}
	\item different layers can even point to different data sets!
\end{itemize}
\end{frame}
\note{so one data set could be raw data, another data set could be summary or confidence interval data}

\begin{frame}[fragile]\frametitle{ggplot - options}
Nicer ggplots
\begin{lstlisting}
ggplot(mydata100, aes(pretest,posttest)) + 
	geom_point() + 
	labs(title = "Plot of Test Scores",
		x = "Before Workshop",
		y = "After Workshop") +
	theme(plot.title = element_text(size=rel(2.5)))
#Colors
library(RColorBrewer)
display.brewer.all(n=4) # how many column patterns
ggplot(mydata100, aes(workshop, fill=gender)) +
	geom_bar(position = "stack") + 
	scale_fill_brewer(palette = "Set1")
\end{lstlisting}
\end{frame}

\begin{frame}[standout]
Exercise 16
\end{frame}

\section{Data description}


\begin{frame}
\frametitle{Frequency tables}
\begin{itemize}
	\item Let $x=\left( x_{1},\ldots ,x_{n}\right) ^{\prime }$ be a vector of
	(numerical or character) observations
	\item The command \texttt{table(x)} returns an object of the class
	\textquotedblleft table\textquotedblright , representing the frequency
	distribution of $x$
	\item The top row shows the values that occur (as a character vector)
	\item The bottom row shows the absolute frequencies
	\item The distribution can be plotted by \texttt{plot(table(x))} or \texttt{%
		barplot(table(x))}
	\item try also \texttt{prop.table(table(x))} for relative frequencies or \texttt{cumsum()prop.table(table(x)))}
\end{itemize}
\end{frame}

\begin{frame}[fragile] \frametitle{Frequency tables}
Examples
\begin{lstlisting}
myWG <- table(workshop,gender)
myWG
summary(myWG)
chisq.test(myWG)
\end{lstlisting}

\end{frame}
\note{we can not reject the null that workshop and gender are independent}


\begin{frame}
\frametitle{Quantiles}
\begin{itemize}
\item Quantiles can be computed by \texttt{quantile(x,prob=...)}
\item The argument \texttt{prob} can be a scalar or a vector of probabilities
\item If \texttt{prob} is a vector the \texttt{quantile} function returns a
vector
\item Note that there are many definitions of quantiles \newline
(see the option \texttt{type} of \texttt{quantile})
\item For large datasets, the differences are negligible
\end{itemize}
\end{frame}


\begin{frame}
\frametitle{Boxplots}
\begin{itemize}
\item If the argument of \texttt{boxplot} is a vector, one boxplot is generated
\item If the argument is a matrix (or data frame), one boxplot for each
column is generated
\end{itemize}
\begin{center}
\includegraphics[width=8cm]{boxplot.jpeg}
\end{center}
\end{frame}


\begin{frame}
\frametitle{Mean, variance, standard deviation}
\begin{description}
\item[\texttt{mean}] Calculates the mean of a vector, the mean of all
elements of a matrix, each column mean of a dataframe
\item[\texttt{sd}] Calculates the standard deviation of a mean, or the
standard deviation of each column of a matrix or dataframe
\item[\texttt{var}] Calculates the variance of a vector, or the covariance
matrix of a matrix or dataframe
\item[\texttt{na.rm}] All three functions have the option \texttt{na.rm}
(remove missings) which can be \texttt{TRUE} or \texttt{FALSE} (default)
\end{description}
\end{frame}


\begin{frame}
\frametitle{Histograms}
\begin{itemize}
\item The built-in command \texttt{hist} generates a plot of the histogram
\item An improved command in the \texttt{library(MASS)} is \texttt{truehist}
\item See the help file of \texttt{truehist} for the options
\item Important options are: \texttt{xlab,ylab,xlim,ylim,main}
\item One can easily add lines and curves to the plot\newline
(with \texttt{abline} or \texttt{lines})
\end{itemize}
\end{frame}


\begin{frame}[fragile]
\frametitle{Histograms}
Examples
\begin{lstlisting}
library(MASS)
x <- rnorm(2000) # random data
truehist(x)
abline(v=0)
g <- seq(-3,3,length=500)
lines(g,dnorm(g))
\end{lstlisting}
\end{frame}

\begin{frame}[standout]
Exercise 17
\end{frame}

\begin{frame}
\frametitle{Covariance}
\begin{itemize}
\item If there are two vectors $x$ and $y$ of the same length $n$, then 
\texttt{cov(x,y)} or \texttt{var(x,y)} compute the covariance
\item If $x$ is a $\left( n,m\right) $-matrix or dataframe, then \texttt{cov(x)} or \texttt{var(x)} compute the covariance matrix of its columns
\item If missing values exist, one can specify which observations should be included (option \texttt{use})
\end{itemize}
\end{frame}


\begin{frame}
\frametitle{Correlation}
\begin{itemize}
\item If there are two vectors $x$ and $y$ of the same length $n$, then 
\texttt{cor(x,y)} computes the correlation coefficient (Bravais-Pearson)
\item If $x$ is a $\left( n,m\right) $-matrix or dataframe, then \texttt{cor(x)} computes the correlation matrix of its columns
\item If missing values exist, one can specify which observations should be included (option \texttt{use})
\item Use the option \texttt{method} to compute Spearman's or Kendall's correlation coefficients
\end{itemize}
\end{frame}

\begin{frame}[fragile]\frametitle{Correlation}
Testing significance
\begin{lstlisting}
cor(select(mydata100,q1:q4),
method="pearson",
use   ="pairwise")

cor.test(mydata$q1,mydata$q4,use="pairwise")
\end{lstlisting}
\end{frame}
\note{first p-values, second p-values are adjusted corrected for the number of tests done.}

\begin{frame}[standout]
Exercise 18
\end{frame}
\section{Cleaning Data}
\begin{frame}[fragile]\frametitle{Missing Values}
\begin{itemize}
	\item Missing values are neither negative nor positive infinity like in STATA
	\item \lstinline|Inf| is infinity, also a kind of missing value, and you CAN do size comparison to it
	\item Finding missing values:
	\begin{itemize}
		\item Not \lstinline|x == NA| but \lstinline|is.na(x)|
		\item Counting missing values:
		\begin{lstlisting}
		x <- c(NA,2,NA,2,1)
		length(x)      #number of all variables
		sum(is.na(x))  #number of missing values
		sum(!is.na(x)) #number of valid values
		\end{lstlisting}
		\item Hint: have a look at \lstinline|n.valid()| from the \lstinline|prettyR| package or write your own function:
		\begin{lstlisting}
		n.missing <- function(x){
		sum(is.na(x))
		}
		\end{lstlisting}
	\end{itemize}	
\end{itemize}
\end{frame}
\note{R's logic is to have many tiny pieces, }


\begin{frame}[fragile]\frametitle{Missing Values}
Setting values to missing
\begin{itemize}
\item R reads numeric blanks as missing
\item Remember: when creating factors, non-specified levels will become missing values
\item When reading text files you can specify NA by option\newline
\lstinline|na.string = c(".","99","999")|
\item Better: use conditional transformations:\\ \lstinline|age[age == 999] <- NA|
\end{itemize}
\end{frame}
\note{na.strings affects all columns of your data set, and the missing values might be different for different values, so better use conditional transformations}

\begin{frame}\frametitle{Action on Missing Values}
\begin{itemize}
\item Summary functions return \lstinline|NA| unless \lstinline|na.rm=TRUE|
\item Modeling functions (that accept formula) automatically exclude \lstinline|NA|s
\item Replacing/Imputing missing values
\begin{itemize}
\item \lstinline|VIM| (Visualization and Imputation of Missing Values): useful to find patterns in missing values and visualize them in color maps (\lstinline|colormapMiss()|), bar charts (\lstinline|barMiss()|) and histograms (\lstinline|histMiss()|)	
\item \lstinline|mice| (Multivariate Imputation by Chained Equations): \lstinline|md.pattern()| function also searches for patterns of missing values
\end{itemize}
\end{itemize}
\end{frame}

\begin{frame}[standout]
Exercise 19
\end{frame}

\section{User-defined functions} 
\begin{frame}
\frametitle{User-defined functions}
\begin{itemize}
\item One can define new functions in R
\item Functions are objects of class \texttt{function}
\item Each function has a name, one or more inputs (arguments) \newline
and one output (return)
\item Inputs can be any objects (usually vectors)
\item The function can return only one object (which can be a list)
\item Variables defined within a function are only local
\end{itemize}
\end{frame}
\note{similar to macros}


\begin{frame}
\frametitle{User-defined functions}
Syntax\medskip

\texttt{fn <- function(x,y)\{}

\quad \texttt{block of commands to compute output out}

\quad \texttt{return(out)}

\quad \texttt{\}\bigskip}
\begin{block}{Example}
\texttt{utility <- function(cons,gam)\{}

\quad \texttt{U <- (cons\symbol{94}(1-gam)-1)/(1-gam)}

\quad \texttt{return(U)}

\quad \texttt{\}}
\end{block}
\end{frame}

\begin{frame}[fragile]
\frametitle{User-defined functions}
Example
\begin{lstlisting}
mystats <- function(x) {
	mymean <- mean(x, na.rm=TRUE)
	mysd   <-   sd(x, na.rm=TRUE)
	c(mean=mymean, sd=mysd) #only last thing is remembered
}
mystats(posttest)
mymean #not found
\end{lstlisting}
\begin{itemize}
\item functions return only a single object, the last one created, but can contain many results
\item applying functions by group
\begin{lstlisting}
by(posttest, workshop, mean)
by(posttest, workshop, mystats)
\end{lstlisting}
\end{itemize}
\end{frame}


\begin{frame}[standout]
Exercise 20
\end{frame}







\section{Programming}


\begin{frame}[fragile]
\frametitle{Loops}
\begin{itemize}
\item If the same commands should be executed for different values of some
variable, loops are useful
\item There are three kinds of loops: \texttt{for}, \texttt{while}, \texttt{repeat}
\item By far the most important loop is the \texttt{for}-loop
\item General syntax:
\end{itemize}
\begin{verbatim}
    for([var] in vector) {
        [commands]
    }
\end{verbatim}
\begin{itemize}
\item The commands are executed for each value of \texttt{vector}
\end{itemize}
\end{frame}


\begin{frame}[fragile]
\frametitle{Loops}
Example
\begin{lstlisting}
z <- rep(NA,10)
for(i in 1:10) {
  z[i] <- i^2
}
print(z)
\end{lstlisting}
\end{frame}


\begin{frame}
\frametitle{Loops}
\begin{itemize}
\item Syntax of the \texttt{while}-loop:
\end{itemize}
\begin{center}
\texttt{while([condition]) \{[commands]\}}
\end{center}
\begin{itemize}
\item Syntax of the \texttt{repeat}-loop:
\end{itemize}
\begin{center}
\texttt{repeat \{[commands]\}}
\end{center}
\begin{itemize}
\item The \texttt{repeat}-loop does never stop but can be left using the
command \texttt{break}
\end{itemize}
\end{frame}


\begin{frame}[fragile]
\frametitle{Conditions}
Syntax of the \texttt{if}-command
\begin{verbatim}
    if([condition]) {
       [commands]
    }
\end{verbatim}
\begin{itemize}
\item The condition must not be a vector
(else only its first element is used)
\item If there is just a single command, the brackets can be omitted
\item The opening curly bracket must appear in the same line as the \texttt{%
if}-command
\item It is possible to add \texttt{else \{[commands]\}}
\end{itemize}
\end{frame}


\begin{frame}
\frametitle{Random numbers}
\begin{itemize}
\item A large number of standard distributions is implemented in R
\item There is a common syntax for cdfs, density functions, quantile
functions, and random number generation:
\begin{description}
\item[\texttt{pNAME(x,pars)}] cumulative distribution function at $x$
\item[\texttt{dNAME(x,pars)}] density (or probability) function at $x$
\item[\texttt{qNAME(p,pars)}] quantile function at $p$
\item[\texttt{rNAME(n,pars)}] generate $n$ random draws
\end{description}
\item Here \texttt{NAME} is the abbreviated name of the distribution and 
\texttt{pars} are its parameters
\end{itemize}
\end{frame}


\begin{frame}
\frametitle{Random numbers}
\framesubtitle{Standard distributions}
Some continuous distribution names:
\begin{description}
\item[\texttt{norm}] normal
\item[\texttt{unif}] uniform
\item[\texttt{lnorm}] log-normal
\item[\texttt{exp}] exponential
\item[\texttt{t}] $t$-distribution
\item[\texttt{chisq}] $\chi ^{2}$-distribution
\item[\texttt{F}] $F$-distribution
\end{description}
\end{frame}


\begin{frame}
\frametitle{Random numbers}
\framesubtitle{Standard distributions}
Some discrete distribution names:
\begin{description}
\item[\texttt{binom}] binomial
\item[\texttt{pois}] Poisson
\item[\texttt{geom}] geometric
\item[\texttt{hyper}] hyper-geometric
\item[\texttt{nbinom}] negative binomial
\item[\texttt{multinom}] multinomial
\end{description}
\end{frame}


\begin{frame}
\frametitle{Random numbers}
\framesubtitle{Standard distributions}
\begin{itemize}
\item Define a vector $x$ on an appropriate grid $[a,b]$
\item Plots of cdf and density functions:
\end{itemize}
\begin{center}
\texttt{plot(x,pNAME(x,pars))}

\texttt{plot(x,dNAME(x,pars))}
\end{center}
\begin{itemize}
\item Define a grid vector $p$ on $[0,1]$; plot of quantile function:
\end{itemize}
\begin{center}
\texttt{plot(x,qNAME(p,pars))}
\end{center}
\end{frame}


\begin{frame}[fragile]
\frametitle{Simulations}
Example: Simulate the distribution of the moment estimator of the exponential distribution
\begin{lstlisting}
R <- 10000
Z <- rep(NA,R)
for(r in 1:R) {
  x <- rexp(n=10,rate=0.5)
  Z[r] <- 1/mean(x)
}
truehist(Z)
abline(v=2,col="red")
\end{lstlisting}
\end{frame}

\begin{frame}[standout]
Exercises 21, 22, 23
\end{frame}


\section{Linear regressions}


\begin{frame}
\frametitle{Multiple linear regression}
\begin{itemize}
\item The general syntax of regression models is rather idiosyncratic:
\end{itemize}
\begin{center}
\texttt{a <- lm(formula)}
\end{center}
\begin{itemize}
\item Basic \textquotedblleft formula\textquotedblright\ syntax
\end{itemize}
\begin{center}
\texttt{y \symbol{126} x1 + x2 + \ldots\ + xK}
\end{center}
\begin{itemize}
\item Endogenous variable is on the left of \texttt{\symbol{126}}; exogenous
variables are on the right of \texttt{\symbol{126}}, separated by \texttt{+}
\item In R modeling functions: accept formulas, create model objects, generic functions show more, extractor functions show more

\end{itemize}
\end{frame}


\begin{frame}[fragile]
\frametitle{Multiple linear regression}
Example
\begin{lstlisting}
library(foreign)
x <- read.dta("wave2009.dta")
attach(x)
regr1 <- lm(satisfaction ~ age + netincome + children)
regr1
plot(regr1)
summary(regr1)
names(regr1)
print(unclass(regr1))
\end{lstlisting}
\end{frame}


\begin{frame}
\frametitle{Multiple linear regression}
\begin{itemize}
\item The \texttt{lm}-object is a list containing:
\begin{enumerate}
\item The estimated coefficients $\hat{\beta}$
\item The residuals $\hat{u}_{t}$
\item The fitted values $\hat{y}_{t}$
\item Some other things
\end{enumerate}
\item If \texttt{a} is an \texttt{lm}-object one can access its elements
using\newline
\texttt{coefficients(a)}, \texttt{residuals(a)}, \texttt{fitted.values(a)}
\item Alternatively, one can use the \texttt{\$}-operator: \texttt{%
a\$coefficients}, \texttt{a\$residuals}, \texttt{a\$fitted.values}
\end{itemize}
\end{frame}


\begin{frame}
\frametitle{Multiple linear regression}
Extensions (I):
\begin{itemize}
\item An intercept is added automatically but can be removed: \texttt{lm(y\symbol{126}x1+x2-1)}
\item If the variables are organized in an unattached dataframe \texttt{x}, 
\newline
one can use the syntax: \texttt{lm(formula,data=x)}
\item The formula may contain mathematical functions, e.g. \texttt{lm(log(y)%
\symbol{126}log(x1))}
\item \textbf{Attention}: Squares, sums and differences are not allowed!
\item Use the function \texttt{I()} for squares, sums and differences
\end{itemize}
\end{frame}


\begin{frame}[fragile]
\frametitle{Multiple linear regression}
Extensions (II):
\begin{itemize}
\item Syntax for interaction terms
\end{itemize}
\begin{center}
\texttt{a <- lm(y \symbol{126} x1 + x2 + x1:x2)}
\end{center}
\begin{itemize}
\item Abbreviations:
\begin{itemize}
	\item \lstinline|a <- lm(y ~ x1*x2)|
	\item \lstinline|a <- lm(y ~ (x1+x2)^2)| for all interactions up to 2
\end{itemize}
\item Weights can be added using the option \texttt{weights}
\item One can select a subset of observations using the option \texttt{subset}
\end{itemize}
\end{frame}


\begin{frame}[fragile]
\frametitle{Multiple linear regression}
Extensions (III):
\begin{itemize}
\item The \texttt{lm}-object can be used to add a regression line to a plot:
\begin{lstlisting}
regr <- lm(y~x)
plot(x,y)
abline(regr)
\end{lstlisting}
\item The \texttt{lm}-object can be used for forecasting:
\begin{lstlisting}
regr <- lm(y~x1+x2)
xn <- data.frame(x1=c(...),x2=c(...))
predict(regr,newdata=xn,se.fit=TRUE)
\end{lstlisting}
\end{itemize}
\end{frame}


\begin{frame}[fragile]
\frametitle{Multiple linear regression}
Extensions (IV):
\begin{itemize}
\item Heteroskedasticity consistent standard errors are not reported by default
\item The package \texttt{sandwich} supplies functions for robust standard errors
\item The syntax for robust standard errors is
\end{itemize}
\begin{lstlisting}
coeftest(regr,vcov=vcovHC)
\end{lstlisting}
\end{frame}


\begin{frame}[fragile]
\frametitle{Multiple linear regression}
Putting it all together:
\begin{lstlisting}
regr2 <- lm(satisfaction~age+netincome,data=x)

regr3 <- lm(satisfaction~age+I(age^2))

regr4 <- lm(satisfaction~log(netincome))

regr5 <- lm(satisfaction~gender*marital)

z <- gender=="Female"

regr6 <- lm(satisfaction~log(netincome),subset=z)

coeftest(regr6,vcovHC)

\end{lstlisting}
\end{frame}

\begin{frame}[fragile]\frametitle{Polynomial regression}
Polynomial regression
\begin{lstlisting}
y ~ x + I(x^2) + I(x^3)
y ~ poly(x,3)
\end{lstlisting}
\end{frame}
\note{I means isolates this call}

\begin{frame}[standout]
Exercise 24
\end{frame}
\section{High Quality Output}

\begin{frame}[fragile]
\frametitle{High Quality Output}
\begin{itemize}
\item Paste into word processor
\item Use packages, e.g. xtable and texreg
\item rtf or R2DOCX to write complex Word docs, but hard to set up
\item Reproducable research: knitr and rnotebook
\end{itemize}
\end{frame}

\begin{frame}[fragile]\frametitle{High Quality Output}
Example
\begin{lstlisting}
myM1 <- lm(q4 ~ q1 + q2 + q3, data=mydata100)
myM2 <- lm(q4 ~ q1, data=mydata100)

library("xtable")
print(xtable(myM1),type="html",file="myM1-xtable.doc")

library("texreg")
htmlreg(myM1, single.row=TRUE,file="myM1-htmlreg.doc")
htmlreg(list(myM1,myM2), file="myM1-myM2-htmlreg.doc")

texreg(list(myM1, myM2))
\end{lstlisting}
\end{frame}
\end{document}
